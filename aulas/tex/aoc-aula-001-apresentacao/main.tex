\documentclass{beamer}
%\documentclass[12pt]{beamer}

%\documentclass[handout]{beamer}

% \setbeameroption{show notes}
% \setbeamertemplate{note page}[plain]
	
\usepackage{etex}

\usepackage{pgfpages}

% inclui o modo handout.
\include{seisporpagina}

\include{colors-defs}

\newcommand{\classnumber} {Aula 001}
\newcommand{\classtitle} {Apresenta��o da Disciplina}
\newcommand{\coursename} {Ci�ncia de Computa��o}
\newcommand{\subjectname} {BCC33B - Arquitetura e Organiza��o de Computadores}
\newcommand{\idsubjectcourseinstitution} {BCC33B-BCC-UTFPR}
\newcommand{\universityname}{Universidade Tecnol�gica Federal do Paran� (UTFPR)}
\newcommand{\departmentname}{Departamento de Computa��o (DACOM)}
\newcommand{\citystatecountry}{Campo Mour�o, Paran�, Brasil}
\newcommand{\authorname} {Prof. Rog�rio Aparecido Gon�alves}
\newcommand{\authorcitation} {R. A. GON�ALVES}
\newcommand{\authoremail} {rogerioag@utfpr.edu.br}
\usepackage[utf8]{inputenc}
\usepackage[brazil]{babel}  % idioma

\usepackage{utopia} %font utopia imported

\usepackage{framed}

%\usetheme{Madrid}
% \usecolortheme{goeagles}
%\usecolortheme{default}

%\usetheme{Frankfurt}
\usetheme{Madrid}
%\usecolortheme[rgb={0.23,0.27,0.29}]{structure}
\usecolortheme[rgb={0.21, 0.27, 0.31}]{structure}

%\usecolortheme{crane}
\setbeamercovered{transparent}

\usefonttheme[onlymath]{serif} % fonte modo matematico

\usepackage{graphics}
\usepackage{subfigure}
\usepackage{float}
\usepackage{multirow}
\usepackage{verbatim}
\usepackage{remreset}
\usepackage{comment} % end and begin comment
\usepackage{dtklogos} % for \BibTeX
\usepackage{listings} % display code on slides; don't forget [fragile] option after \begin{frame}
\usepackage{color}
\usepackage{url}
\usepackage{tikz}
\usepackage[tikz]{bclogo}

\usepackage{ragged2e} % justifying.

\usepackage{lstlisting-llvm}

\usepackage{adjustbox} % ajustar codigo.

\usepackage[normalem]{ulem} % tachado.

\usepackage{xkeyval}
\usepackage{todonotes}
\presetkeys{todonotes}{inline}{}


%\usepackage{enumitem} % alterar a distancia dos itens dos itemize.

% Colocar boxes com descricao de figuras.
\usepackage{graphicx}
\newcommand{\putat}[3]{\begin{picture}(0,0)(0,0)\put(#1,#2){#3}\end{picture}}

% Opçõess de listing usados para o codigo fonte
% Ref: http://en.wikibooks.org/wiki/LaTeX/Packages/Listings
\lstset{ %
language=Java,                  % choose the language of the code
basicstyle=\footnotesize,       % the size of the fonts that are used for the code
%basicstyle=\ttfamily,
stringstyle=\ttfamily\color[rgb]{0.16,0.16,0.16},
numbers=left,                   % where to put the line-numbers
numberstyle=\footnotesize,      % the size of the fonts that are used for the line-numbers
stepnumber=1,                   % the step between two line-numbers. If it's 1 each line will be numbered
numbersep=2pt,                  % how far the line-numbers are from the code
showspaces=false,               % show spaces adding particular underscores
showstringspaces=false,         % underline spaces within strings
showtabs=true,					% show tabs within strings adding particular underscores
frame=single,	                % adds a frame around the code
framerule=0.6pt,
tabsize=2,	                	% sets default tabsize to 2 spaces
%keepspaces,					% set one line at code final.
extendedchars=true,
captionpos=b,                   % sets the caption-position to bottom
breaklines=true,                % sets automatic line breaking
breakatwhitespace=false,        % sets if automatic breaks should only happen at whitespace
aboveskip=5pt,
upquote=true,
columns=fixed,
escapeinside={\%*}{*)},         % if you want to add a comment within your code
backgroundcolor=\color[rgb]{1.0,1.0,1.0}, % choose the background color.
rulecolor=\color[rgb]{0.8,0.8,0.8},
xleftmargin=10pt,
xrightmargin=10pt,
framexleftmargin=10pt,
framexrightmargin=10pt
}

%%% RAG----------------------------------------------------------------%
%-----------------------Style Definitions------------------------------%
\definecolor{javared}{rgb}{0.6,0,0} % for strings
\definecolor{javagreen}{rgb}{0.25,0.5,0.35} % comments
\definecolor{javapurple}{rgb}{0.5,0,0.35} % keywords
\definecolor{javadocblue}{rgb}{0.25,0.35,0.75} % javadoc

\definecolor{DarkBlue}{rgb}{0,0,0.61}
\definecolor{DarkGreen}{rgb}{0,0.4,0}
\definecolor{DarkRed}{rgb}{0.51,0,0}
\definecolor{DarkBlue2}{rgb}{0.25,0.51,0}


% Numeros.
\lstdefinestyle{mynumbers}{
	numbers=left,
	stepnumber=1,
	numbersep=4pt,
	numberstyle=\tiny\color{black}
}
% Text Code.
\lstdefinestyle{mytextcode}{
	basicstyle=\footnotesize,
	tabsize=2,
	showspaces=false,
	showstringspaces=false,
	extendedchars=true,
	breaklines=true
}
% Frame.
\lstdefinestyle{myframe}{
	backgroundcolor=\color{white},
	frame=trbl
}
% C++ Style.
\lstdefinestyle{C++}{
	language=C++,
	style=mynumbers,
	style=mytextcode,
	style=myframe,
	keywordstyle=\color{black}\bfseries,
	stringstyle=\color{gray},
	commentstyle=\color[rgb]{0.08,0.08,0.08},
	morecomment=[s][\color{lightgray}]{/*}{*/},
	otherkeywords={dim3},
  	emph={ \_\_device\_\_, \_\_global\_\_, \_\_shared\_\_, \_\_host\_\_, \_\_constant\_\_},
	emphstyle=\color{DarkBlue}\bfseries,
	emph={[2] printf, scanf, \#include, \#define, \#pragma, \#typedef},
	emphstyle=[2]\color{DarkGreen},
}
% C Style.
\lstdefinestyle{C}{
	language=C,
	style=mynumbers,
	style=mytextcode,
	style=myframe,
	basicstyle=\ttfamily,
	keywordstyle=\color{javapurple}\bfseries,
  	stringstyle=\color{gray}\bfseries,
  	commentstyle=\color[rgb]{0.08,0.08,0.08},
  	morecomment=[s][\color{lightgray}]{/*}{*/},
  	otherkeywords={dim3, \#define, \#pragma, \#typedef},
  	emph={ \_\_device\_\_, \_\_global\_\_, \_\_shared\_\_, \_\_host\_\_, \_\_constant\_\_},
  	emphstyle=\color{DarkBlue}\bfseries,
  	emph={[2] printf, scanf, \#include},
  	emphstyle=[2]\color{DarkGreen},
  	emph={[3] omp, oac},
  	emphstyle=[3]\color{DarkRed},
  	emph={[4] single, parallel, runtime},
  	emphstyle=[4]\color{DarkBlue2},
	backgroundcolor={},
	identifierstyle=\color{black}	
}
% Bash Style.
\lstdefinestyle{bash}{
	language=bash,
	style=mynumbers,
	style=mytextcode,
	style=myframe,
	backgroundcolor={},
	frame=single,
	basicstyle=\scriptsize\ttfamily
}
% Python Style.
\lstdefinestyle{python}{
	language=python,
	style=mynumbers,
	style=mytextcode,
	style=myframe,
	backgroundcolor={}
}
% Java Style.
\lstdefinestyle{java}{
	language=java,
	style=mynumbers,
	style=mytextcode,
	style=myframe,
	backgroundcolor={},
	basicstyle=\ttfamily,
	keywordstyle=\color{javapurple}\bfseries,
	stringstyle=\color{javared},
	commentstyle=\color{javagreen},
	morecomment=[s][\color{javadocblue}]{/**}{*/}
}
% ASM Style.
\lstdefinestyle{asm}{
  %belowcaptionskip=1\baselineskip,
  %xleftmargin=\parindent,
  language=[x86masm]Assembler,
  style=mynumbers,
  style=mytextcode,
  style=myframe,
  backgroundcolor={},
  frame=single,
  basicstyle=\scriptsize\ttfamily,
  keywordstyle=\color{javapurple}\bfseries,
  stringstyle=\color{gray}\bfseries,
  commentstyle=\itshape\color{red!40!gray},
  identifierstyle=\color{black},
  otherkeywords={movl, leaq, movq, subq, jmp, jg, pushq, popq, addl, cmpl, movss, cmpb, idivl, cltd},
  emph={eax, ebx, ecx, edx, esi, edi, ebp, eip, esp, r8d, r9d, r10d, r11d, r12d, r13d, r14d, r15d},
  emphstyle=\color{DarkGreen}\bfseries,
  emph={[2]rax, rbx, rcx, rdx, rsi, rdi, rbp, rip, rsp, r8, r9, r10, r11, r12, r13, r14, r15},
  emphstyle=[2]\color{DarkRed}\bfseries  
}

% Fortran Style.
\lstdefinestyle{fortran}{
  language=[90]Fortran,
  style=mynumbers,
  style=mytextcode,
  style=myframe,
  backgroundcolor={},
  frame=single,
  basicstyle=\footnotesize,
  commentstyle=\itshape\color{purple!40!black},
  morecomment=[l]{!\ }% Comment only with space after !
}

% LLVM Style.
\lstdefinestyle{llvm}{
	language=llvm,
	%inputencoding=utf8,
	style=mynumbers,
	style=mytextcode,
	style=myframe,
	backgroundcolor={},
	frame=single,
	basicstyle=\scriptsize\ttfamily,
  tabsize=4,
  %rulecolor=,
  upquote=true,
% aboveskip={1.5\baselineskip},
  columns=fixed,
  prebreak = \raisebox{0ex}[0ex][0ex]{\ensuremath{\hookleftarrow}},
  showtabs=false,
	%basicstyle=\scriptsize\upshape\ttfamily,
  identifierstyle=\ttfamily,
  keywordstyle=\ttfamily\bfseries\color[rgb]{0,0,0},
  %commentstyle=\ttfamily\color[rgb]{0.133,0.545,0.133},
  commentstyle=\ttfamily\color[rgb]{0.08,0.08,0.08},
  %stringstyle=\ttfamily\color[rgb]{0.627,0.126,0.941}
  stringstyle=\ttfamily\color[rgb]{0.16,0.16,0.16},
  morecomment = [l]{;},
    morestring=[b]", 
    sensitive = true,
    classoffset=0,
    morekeywords={
      define, declare, global, constant,
      internal, external, private,
      linkonce, linkonce_odr, weak, weak_odr, appending,
      common, extern_weak,
      thread_local, dllimport, dllexport,
      hidden, protected, default,
      except, deplibs,
      volatile, fastcc, coldcc, cc, ccc,
      x86_stdcallcc, x86_fastcallcc,
      ptx_kernel, ptx_device,
      signext, zeroext, inreg, sret, nounwind, noreturn,
      nocapture, byval, nest, readnone, readonly, noalias, uwtable,
      inlinehint, noinline, alwaysinline, optsize, ssp, sspreq,
      noredzone, noimplicitfloat, naked, alignstack,
      module, asm, align, tail, to,
      addrspace, section, alias, sideeffect, c, gc,
      target, datalayout, triple,
      blockaddress
    },
    classoffset=1, keywordstyle=\color{purple},
    morekeywords={
      fadd, sub, fsub, mul, fmul,
      sdiv, udiv, fdiv, srem, urem, frem,
      and, or, xor,
      icmp, fcmp,
      eq, ne, ugt, uge, ult, ule, sgt, sge, slt, sle,
      oeq, ogt, oge, olt, ole, one, ord, ueq, ugt, uge,
      ult, ule, une, uno,
      nuw, nsw, exact, inbounds,
      phi, call, select, shl, lshr, ashr, va_arg,
      trunc, zext, sext,
      fptrunc, fpext, fptoui, fptosi, uitofp, sitofp,
      ptrtoint, inttoptr, bitcast,
      ret, br, indirectbr, switch, invoke, unwind, unreachable,
      malloc, alloca, free, load, store, getelementptr,
      extractelement, insertelement, shufflevector,
      extractvalue, insertvalue,
    },
    alsoletter={\%},
    keywordsprefix={\%},
}
%%% RAG----------------------------------------------------------------%

\renewcommand{\lstlistingname}{Código}

% \newcommand{\putat}[3]{\begin{picture}(0,0)(0,0)\put(#1,#2){#3}\end{picture}}

\setbeamertemplate{caption}[numbered]


% % % RAG -------------------------------------------------------------- %
% References.
\usepackage{natbib}

% % % RAG -------------------------------------------------------------- %
% Caixas
\usepackage[most]{tcolorbox}
\usepackage{lipsum}

\usepackage{algorithm,algorithmic}
 

\usepackage{appendixnumberbeamer}

\newtcolorbox{mybox}[3][]
{
	breakable,
	boxsep=1pt,
	left=2pt,
	right=2pt,
	top=1pt,
	bottom=1pt,
	colframe = #2!40!black,
	colback  = #2!10,
	% coltitle = #2!20!black,  
	title    = #3,
	#1,
}

%-----------------------------------------------------
%This block of code defines the information to appear in the
%Title page
\title[\idsubjectcourseinstitution] %optional
{\texttt{\classnumber}: \classtitle}
%\subtitle{Visão Geral}
\normalsize
\author[\authorcitation] % (optional)
{\authorname\inst{1} \\
	\texttt{\authoremail}
}

\normalsize
\institute[] % (optional)
{
  \inst{1}%
  \universityname \\
  \departmentname \\
  \citystatecountry
  \and
  \large
  \textbf{\coursename}
  
  \vspace{-1.0cm}
}
\date[\today] % (optional)
{\vspace{-1.5cm} \texttt{\subjectname}}

% \logo{\includegraphics[height=0.7cm]{logos/logos.png}}
%\titlegraphic{\includegraphics[width=2cm]{logos/logos.png}\hspace*{4.75cm}~%
%	\includegraphics[width=2cm]{logos/logos.png}
%}

% \titlegraphic{\includegraphics[height=0.7cm]{logos/logos.png}\vspace*{-0.2cm}}

\setbeamertemplate{navigation symbols}{}

%End of title page configuration block
%-------------------------------------------------------

%-----------------------------------------------------
%The next block of commands puts the table of contents at the 
%beginning of each section and highlights the current section:

%\AtBeginSection[]
%{
%  \begin{frame}
%    \frametitle{Agenda}
%    \tableofcontents[currentsection]
%  \end{frame}
%}
%-----------------------------------------------------
\begin{document}

%The next statement creates the title page.
%\frame{\titlepage}
%\note{note}

{
	\setbeamertemplate{headline}{}
	\usebackgroundtemplate{\includegraphics[trim=1.0cm 1.0cm 1.0cm 1.0cm, scale=0.47]{logos/header-utfpr.pdf}}
	\begin{frame}
		\maketitle
	\end{frame}
}

%-----------------------------------------------------
%-----------------------------------------------------
%This block of code is for the table of contents after
%the title page
\begin{frame}
\frametitle{Agenda}
\tableofcontents
\end{frame}
%-----------------------------------------------------
\section[Introdução]{Introdução}
%-----------------------------------------------------
\begin{frame}
\frametitle{Objetivos}
\justifying
\Large
\begin{itemize}
    \item Uma visão geral sobre a disciplina.
    \item Apresentar os critérios de avaliação.
    \item Plano de Ensino.
\end{itemize}
\end{frame}
%-----------------------------------------------------
\section[Disciplina]{Disciplina}
%-----------------------------------------------------
\begin{frame}
	\frametitle{BCC33B - Arquitetura e Organização de Computadores}
	\fontsize{14pt}{7.2}\selectfont
	\begin{minipage}[t][\textheight][t]{\textwidth}
	% \begin{itemize}
		% \item  \textbf{BCC33B - Arquitetura e Organização de Computadores}
		% \item \textbf{Objetivos:} Compreender os conceitos fundamentais de Arquitetura e Organização de Computadores, o funcionamento de computadores através do estudo do ciclo de instrução e do tráfego de informações dentro da CPU e até a memória e unidades de entrada e saída. 
		% \item \textbf{Ementa:} Aritmética para computadores com inteiros e ponto flutuante. Arquiteturas gerais de computadores. Arquiteturas RISC e CISC. CPU. ALU. Instruções e linguagem de máquina. Modos de endereçamento. Sistemas de memória cache, virtual, principal e externa. Pipeline. Mecanismos de interrupção. Interface com periféricos. Arquiteturas Paralelas e não Convencionais.
	% \end{itemize}
	\begin{exampleblock}{Objetivos}
		\fontsize{12pt}{7.2}\selectfont
		Compreender os conceitos fundamentais de Arquitetura e Organização de Computadores, o funcionamento de computadores através do estudo do ciclo de instrução e do tráfego de informações dentro da CPU e até a memória e unidades de entrada e saída.
	\end{exampleblock}
	\begin{alertblock}{Ementa}
		\fontsize{12pt}{7.2}\selectfont
		Aritmética para computadores com inteiros e ponto flutuante. Arquiteturas gerais de computadores. Arquiteturas RISC e CISC. CPU. ALU. Instruções e linguagem de máquina. Modos de endereçamento. Sistemas de memória cache, virtual, principal e externa. Pipeline. Mecanismos de interrupção. Interface com periféricos. Arquiteturas Paralelas e não Convencionais.
	\end{alertblock}
	\end{minipage}
\end{frame}
%-----------------------------------------------------
\begin{frame}[fragile]
	\frametitle{Arquitetura e Organização de Computadores}
	\fontsize{14pt}{7.2}\selectfont
	\begin{minipage}[t][\textheight][t]{\textwidth}
		\begin{itemize}
			\item  É uma disciplina básica para a Computação.
			\item Cursos relacionados à computação necessitam de disciplinas dessa natureza para sua fundamentação conceitual.
			\item \textbf{Na Computação:} Ingresso na pós-graduação: \texttt{POSCOMP}
			\item \textbf{Para vocês:} Contribuição para o entendimento de Sistemas de Computação.
			\item Entender como o Computador funciona é indispensável a qualquer profissional da área, mesmo que não atue diretamente neste assunto.
		\end{itemize}
	\end{minipage}
\end{frame}
%-----------------------------------------------------
\begin{frame}
	\frametitle{Relação disciplina e curso}
	\begin{figure}[htbp]
		\includegraphics[width=\textwidth,height=\textheight]{figures/grade-bcc.png}
	\end{figure}
\end{frame}
%-----------------------------------------------------------------------
\begin{frame}
	\frametitle{O que deve estar claro}
	\fontsize{14pt}{7.2}\selectfont
	\begin{minipage}[t][\textheight][t]{\textwidth}
	\begin{itemize}
		\item Resolver as listas de exercícios é uma boa forma de preparar-se para a avaliação.
		\item Revisar a matéria semanalmente para não acumular conteúdo também é uma abordagem interessante.
		\item Façam as atividades e participem das aulas.
		\item Reclamar na véspera da avaliação pela quantidade de matéria acumulada para estudar, também não vale!
		\item Se dediquem e NÃO deixem para estudar na última hora.
	\end{itemize}
	\begin{figure}[htbp]
		\includegraphics[scale=0.2]{figures/estudando-na-moto.jpg}
	\end{figure}
	\end{minipage}
\end{frame}
%-----------------------------------------------------
\section[Ensino]{Ensino}
%-----------------------------------------------------
%\begin{frame}
%\frametitle{Plano de Ensino}
%\begin{itemize}
%    \item Datas de avaliações e trabalhos:
%    \begin{itemize}
%        \item {\textbf{Projeto A} (entrega/apresentação) - 17/12}
%        \item {\textbf{1ª Avaliação} - 18/12}
%        \item {\textbf{Projeto B} (entrega/apresentação) - 11/03}
%        \item {\textbf{2ª Avaliação} - 12/03}
%    \end{itemize}
%    \item Critério:
%    \begin{itemize}
%        \item{\textbf{Avaliação 1 (AV1)}}
%        \[ AV1 = P1 * 0,5 + PA * 0,3 \]
%        \item{\textbf{Avaliação 2 (AV2)}}
%        \[ AV2 = P2 *  0,5 + PB * 0,3 \]
%        \item A nota final será calculada pela fórmula:
%        \[ NF = \frac{(AV1 + AV2)}{2} + (ACA * 0,2) \]
%        \item Avaliação Final (AF)
%        \[ RF = \frac{(NF + AF)}{2} \geq 6,0 \]
%    \end{itemize}
%\end{itemize}
%\end{frame}
%-----------------------------------------------------------------------
\begin{frame}[allowframebreaks]
	\frametitle{Plano de Ensino}
	\fontsize{14pt}{7.2}\selectfont
	\begin{minipage}[t][\textheight][t]{\textwidth}
	\vspace{-0.5cm}
	\begin{columns}
		\column{0.5\textwidth}
			\begin{figure}
				\centering
				\includegraphics[page=1,trim=2.0cm 3.0cm 2.0cm 3.0cm, scale=0.3]{figures/plano-ensino-BCC33B-IC3A}
			\end{figure}
		\column{0.5\textwidth}
			\begin{figure}
				\centering
				\includegraphics[page=2,trim=2.0cm 3.0cm 2.0cm 3.0cm, scale=0.3]{figures/plano-ensino-BCC33B-IC3A}
			\end{figure}
	\end{columns}
	\end{minipage}
\end{frame}
%-----------------------------------------------------------------------
\begin{frame}[allowframebreaks]
	\frametitle{Plano de Ensino}
	\fontsize{14pt}{7.2}\selectfont
	\begin{minipage}[t][\textheight][t]{\textwidth}
		\vspace{-0.5cm}
		\begin{columns}
			\column{0.5\textwidth}
			\begin{figure}
				\centering
				\includegraphics[page=3,trim=2.0cm 3.0cm 2.0cm 3.0cm, scale=0.3]{figures/plano-ensino-BCC33B-IC3A}
			\end{figure}
			\column{0.5\textwidth}
			\begin{figure}
				\centering
				\includegraphics[page=4,trim=2.0cm 3.0cm 2.0cm 3.0cm, scale=0.3]{figures/plano-ensino-BCC33B-IC3A}
			\end{figure}
		\end{columns}
	\end{minipage}
\end{frame}
%-----------------------------------------------------------------------
%\begin{frame}
%	\frametitle{Material Adicional}
%	\fontsize{14pt}{7.2}\selectfont
%	\begin{minipage}[t][\textheight][t]{\textwidth}
%	\begin{itemize}
%		\item  Livro gratuito do Prof. Carlos Maziero (UTFPR).
%		\item  \url{http://dainf.ct.utfpr.edu.br/~maziero}
%		\item  \url{http://dainf.ct.utfpr.edu.br/~maziero/doku.php/so:livro_de_sistemas_operacionais}
%	\end{itemize}
%	\end{minipage}
%\end{frame}
%-----------------------------------------------------------------------
\section[Metodologia]{Metodologia}
\begin{frame}
	\frametitle{Metodologia de Ensino}
	\fontsize{14pt}{7.2}\selectfont
	\begin{minipage}[t][\textheight][t]{\textwidth}
		\begin{itemize}
			\item  Aulas expositivas e práticas.
			\item Teoria (Conceitos e Exemplos) + Exercícios.
			\item Resumo da aula (Individual).
			\item Revisão (simplificada) da aula anterior no início de cada aula.
			\item Missão Im\textbf{possível}: Convencer os alunos a revisarem o conteúdo semanalmente.
			\item \textbf{Tarefa dos alunos:} Estudar cada aula para não acumular o conteúdo e para dar bons exemplos a outros colegas.
		\end{itemize}
	\end{minipage}
\end{frame}
%-----------------------------------------------------------------------
\section[Atendimento]{Atendimento}
\begin{frame}
	\frametitle{Professor}
	\fontsize{14pt}{7.2}\selectfont
	\begin{minipage}[t][\textheight][t]{\textwidth}
		\vspace{1.0cm}
	\begin{center}
		\textbf{Rogério Aparecido Gonçalves (RAG)} \\
		%            Doutorando em Ciência da Computação – IME/USP (2011 ...) \\
		%            Mestre em Ciência da Computação – PCC/UEM  (2008) \\
		%            Bacharel em Informática – UEM (2005)  \\
		\texttt{rogerioag@utfpr.edu.br}
	\end{center}
	\begin{itemize}
		\item Professor na UTFPR desde de JAN/2009.
		\item Estive afastado para Doutorado entre SET/2014 e MAI/2016.
		\item Áreas:
		\begin{itemize}
			\fontsize{14pt}{7.2}\selectfont
			\item Computação Paralela em Sistemas Heterogêneos
			\item Arquitetura de Computadores e Processamento Paralelo
			\item Compiladores e \emph{Runtimes}
		\end{itemize}
	\end{itemize}
	\end{minipage}
\end{frame}
%-----------------------------------------------------------------------
\begin{frame}
	\frametitle{Horário}
	\fontsize{14pt}{7.2}\selectfont
	\begin{minipage}[t][\textheight][t]{\textwidth}
		\begin{figure}
			\centering
			\includegraphics[trim=1.0cm 5.0cm 1.0cm 16.3cm, clip, width=0.9\textwidth]{figures/horario-rag-2017-01}
		\end{figure}
		\begin{itemize}
			\item \textbf{Atendimento:} QUI.
			\item \textbf{Obs:} Horários de Atendimento são para tirar dúvidas no decorrer das aulas e não somente em véspera de Prova.
			\item Utilizem os recursos colocados à disposição de vocês!
		\end{itemize}
	\end{minipage}
\end{frame}
%-----------------------------------------------------------------------
\begin{frame}
	\frametitle{Canais de Comunicação}
	\fontsize{14pt}{7.2}\selectfont
	\begin{minipage}[t][\textheight][t]{\textwidth}
		\begin{itemize}
			\item \textbf{Departamento:} DACOM
			\item \textbf{Coordenação:} COINT/COCIC
			\item \textbf{Sala:} E-006/12
			\item Horários de atendimento
			\item \textbf{E-mail institucional:} \texttt{rogerioag@utfpr.edu.br}
			\item \textbf{Moodle:} Informações, materiais e avaliações \url{http://moodle.utfpr.edu.br}
			\item E-mail da turma? Redes Sociais...
		\end{itemize}
	\end{minipage}
\end{frame}
%-----------------------------------------------------------------------
\section[Convivência]{Convivência}
\begin{frame}
	\frametitle{Regimento disciplinar}
	\fontsize{14pt}{7.2}\selectfont
	\begin{minipage}[t][\textheight][t]{\textwidth}
		\begin{itemize}
			\item Leiam o Regimento Disciplinar
			\item {Disponível em: \url{http://www.cm.utfpr.edu.br/images/arquivospdf/regdisciplinar.pdf}}
			\item É importante saber sobre os Direitos e Deveres.
		\end{itemize}
	\end{minipage}
\end{frame}
%-----------------------------------------------------------------------
\begin{frame}
	\frametitle{Boa convivência}
	\fontsize{14pt}{7.2}\selectfont
	\begin{minipage}[t][\textheight][t]{\textwidth}
		\begin{itemize}
			\item  Horário
			\item  Uso de celulares, notebook e outros
			\item  Dediquem-se agora
			\item  Procuro ser justo: \texttt{(trabalho = crédito) -> resultado}
			\item  \textbf{Plágio}
			Segundo Art. 9o., as sanções disciplinares cabíveis são:
			\begin{enumerate}[I.]
				\item advertência escrita;
				\item suspensão;
				\item expulsão.
			\end{enumerate}
		\end{itemize}
	\end{minipage}
\end{frame}
%-----------------------------------------------------------------------
\section[Dúvidas]{Dúvidas}
\begin{frame}
	\frametitle{Perguntas?}
	\fontsize{14pt}{7.2}\selectfont
	\begin{minipage}[t][\textheight][t]{\textwidth}
		
	\end{minipage}
\end{frame}
%-----------------------------------------------------
\section[Referências]{Referências}
%-----------------------------------------------------
\begin{frame}[t,allowframebreaks]
	\frametitle{Referências}
    \scriptsize
    \raggedright
    % \nocite{*}
    % \bibliographystyle{amsalpha}
    \bibliographystyle{apalike}
    % make bibliography entries smaller
    % \renewcommand\bibfont{\scriptsize}
    % If you have more than one page of references, you want to tell beamer
    % to put the continuation section label from the second slide onwards
    \setbeamertemplate{frametitle continuation}[from second]
    % Now get rid of all the colours
    \setbeamercolor*{bibliography entry title}{fg=black}
    \setbeamercolor*{bibliography entry author}{fg=black}
    \setbeamercolor*{bibliography entry location}{fg=black}
    \setbeamercolor*{bibliography entry note}{fg=black}
    % and kill the abominable icon
    \setbeamertemplate{bibliography item}{}
        
    \bibliography{ref}
\end{frame}
%-----------------------------------------------------

\appendix
%-----------------------------------------------------
\begin{frame}[fragile, allowframebreaks]
	\frametitle{Extras}
	\begin{minipage}[t][\textheight][t]{\textwidth}
	\end{minipage}
\end{frame}
%-----------------------------------------------------
\end{document}
